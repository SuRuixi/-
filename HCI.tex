\documentclass{article}
% \usepackage[utf8]{inputenc}
\usepackage{ctex}
\usepackage{geometry}
\usepackage{hyperref} % 加载 hyperref 包
\geometry{
	a4paper,
	total={170mm,257mm},
	left=20mm,
	top=20mm,
}
\usepackage{graphicx}
\usepackage{titling}

\title{Malicious Selling Strategies in Livestream E-commerce:
	\\A Case Study of Alibaba's Taobao and ByteDance's TikTok\\{\Large 恶意销售策
		略
		在直播电
		商中的应用:
		以淘宝和抖音为例的研究}}

\date{2024年12月}

\usepackage{fancyhdr}
\fancypagestyle{plain}{%  the preset of fancyhdr 
	\fancyhf{} % clear all header and footer fields
	\fancyfoot[L]{\thedate}
	\fancyhead[L]{中山大学人机交互课论文研读报告}
}
\makeatletter
\def\@maketitle{%
	\newpage
	\null
	\vskip 1em%
	\begin{center}%
		\let \footnote \thanks
		{\LARGE \@title \par}%
		\vskip 1em%
		%{\large \@date}%
	\end{center}%
	\par
	\vskip 1em}
\makeatother

\usepackage{lipsum}  
\usepackage{cmbright}

\begin{document}
	
	\maketitle
	
	\textbf{学号}: 22330100
	
	\textbf{论文原作者}: QUNFANG WU and YISI SANG, 
	DAKUO WANG, 
	ZHICONG LU
	
	\textbf{论文出处}: The 2023 ACM Transactions on Computer-Human Interaction 
	(TOCHI) , \href{https://dl.acm.org/doi/10.1145/3577199}{paper's DOI}
	
	\section*{研究背景和动机}
近年来,随着电子商务的迅猛发展,直播购物已成为一种新兴的在线消费形式,尤其在新冠疫情的背
景下,其普及速度显著加快。传统的电视购物模式被赋予了互动性更强、用户参与度更高的特点,直
播购物平台(如淘宝和抖音)通过整合即时购买功能和社交互动机制,为消费者提供了便捷且充满吸
引力的购物体验。然而,这一新兴模式的快速扩展也带来了问题,其中直播中恶意销售行为的增加尤
其引发关注。这些行为包括虚假宣传、强制购买等,通过不对称的信息与权力关系,误导甚至强迫消
费者做出不符合其利益的购买决策。

尽管已有研究探讨了消费者和直播主在直播购物情境下的行为动机及其信任构建的机制,但对直播购
物中恶意销售策略及其对消费者的负面影响的关注仍较为有限。同时,平台设计在这些行为中所扮演
的中介角色鲜有被深入分析。现有研究多集中于传统线上购物或社交媒体中的欺诈性设计,但直播购
物独特的互动性和实时性可能为恶意行为的实施提供了新的途径和挑战。因此,明确和分类这些恶意
销售策略,并探讨平台设计如何放大或限制这些行为,对于保护消费者权益和优化平台设计具有重要
的理论与实践意义。

基于此,本文提出三大研究问题:(1)直播购物中存在哪些恶意销售策略?(2)平台设计如何影响
这些策略的实施?(3)消费者如何感知并应对这些策略?为回答这些问题,作者通过对淘宝和抖音
平台上40场直播购物会话的质性分析,系统识别了16种恶意销售策略,并结合访谈揭示了消费者在
面对这些策略时的态度、应对方式及所面临的挑战。研究结果不仅为理解直播购物中的负面行为提供
了新的视角,也为平台政策制定和用户界面设计提供了重要启示。
	
	\section*{研究方法和结果}
本文采用两阶段研究方法以深入探讨直播购物中的恶意销售策略及其影响机制。第一阶段,研究者通
过质性内容分析,对来自淘宝和抖音两大直播购物平台的40场公开直播会话进行了系统记录与编码。
这些会话涵盖多种产品类别(如食品、美妆、服装等)及不同规模的主播,旨在捕捉恶意销售策略的
多样性与普遍性。研究团队结合现有文献中的分类框架,首先进行演绎编码以识别已知的恶意策略,
随后通过归纳分析扩展并完善了新的策略分类。第二阶段,研究者开展了半结构化访谈,与13位拥有
直播购物经验的用户深入讨论其对这些策略的感知、态度及应对方式。访谈数据以开放编码和轴心编
码方法进行分析,从而形成关于消费者感知与反应的综合性视角。

研究结果揭示了直播购物中的16种恶意销售策略,并将其归纳为四类:限制性策略、欺骗性策略、隐
蔽性策略和不对称性策略。这些策略涵盖了虚假稀缺性、虚假紧迫性、强制订阅、误导性视觉表现等
典型手段。研究发现,其中九种策略被平台设计放大或支持,例如通过提醒标签、计时器、社交证据
通知等功能强化了主播的操控能力。此外,访谈数据显示,大多数消费者对部分明显的策略(如虚假
稀缺性)有一定感知,但对于隐蔽性和不对称性策略的意识较为薄弱。此外,消费者即使识别出这些
策略,通常仍面临权力不平衡、信息不对称以及平台设计限制等挑战,使得有效抵御这些恶意行为变
得困难。

通过揭示恶意销售策略的分类及其平台设计的中介作用,本文为理解直播购物的负面机制提供了重要
贡献,并进一步提出了针对平台设计与政策改进的建议。这些发现不仅拓展了人机交互领域中关于欺
骗性设计和用户决策影响的理论框架,也为实践中优化直播电商平台设计、平衡商家与消费者的利益
提供了实用性的指导。
	
	
	\section*{批判性思考}
这篇论文以严谨的研究设计和细致的实证分析,为理解直播购物中的恶意销售行为提供了有价值的理
论与实践洞见。然而,作为一项学术研究,其仍存在一些可以进一步讨论和改进的方面。

首先,在研究范围上,尽管本文聚焦于中国两大主流直播购物平台——淘宝与抖音——其代表性较强,
但未能充分考虑其他国际化平台(如亚马逊直播或Facebook Live)可能的文化与机制差异。这种局
限性可能影响研究结论在不同国家和文化背景下的普适性。未来研究可以通过跨平台的比较分析,探
讨不同文化背景和平台设计对恶意销售策略的差异化影响,从而扩展结论的外部有效性。

其次,研究方法虽然系统且严谨,但也存在一些可能的局限。例如,本文的数据主要来源于公开直播
会话和访谈,而忽略了与平台开发者或主播本人的对话,这可能导致对平台设计初衷和主播动机的理
解不足。此外,研究对恶意策略的定义和分类在一定程度上依赖研究者的主观判断,尽管团队通过多
次讨论和验证以提高可靠性,但仍难以完全避免编码过程中的潜在偏差。未来研究可以通过引入量化
分析或实验设计,更系统地验证这些策略对消费者行为的具体影响。

再次,论文在讨论平台设计对恶意销售策略的中介作用时,提出了一些建设性建议,但未能充分考虑
平台运营方和主播的利益与动力。直播购物平台在一定程度上依赖于主播的吸引力与销售能力来维持
自身的商业模式,而论文建议的设计改进可能会削弱主播的盈利能力,从而影响平台生态的平衡。对
此,未来研究可以进一步探索在保护消费者权益的同时,如何通过激励机制或透明化规则实现利益相
关者之间的平衡。

此外,论文未能深入讨论消费者对恶意销售策略的长期适应性问题。访谈结果揭示了一些消费者在面
对恶意行为时的应对策略,但缺乏对消费者长期适应过程的动态性分析。例如,消费者在初次识别恶
意行为后是否会改变消费行为或选择其他平台,这些行为变化可能对直播电商生态系统的稳定性产生
深远影响。这种动态性分析可以通过纵向研究加以探讨,从而更全面地揭示消费者行为与平台设计之
间的复杂互动关系。

最后,论文中关于政策与设计改进的讨论虽然提供了良好的实践建议,但缺乏对其实施难度和潜在副
作用的评估。例如,限制某些设计功能可能会导致平台互动性下降或用户体验的复杂化,而过于严格
的政策可能抑制平台创新。未来研究可以结合实地实验或与平台方的合作,验证这些改进措施的实际
效果和可行性,并提出更平衡的解决方案。

总的来说,这篇论文在揭示直播购物恶意销售策略及其机制方面具有重要的理论贡献,并为政策制定
者和平台设计者提供了宝贵的实践指导。然而,其在研究范围、方法选择及实践建议的深度上仍有改
进空间。通过进一步扩展跨文化研究、加强定量验证、关注动态适应性以及评估设计优化的可行性,
未来研究可以更全面地揭示直播购物的复杂性及其优化路径。

\newpage

\emph{{	[请在这里写跟这篇文章相关的你的深入的批判性思考,例如:检查论文中使用的证据
是否充分
	、相关和可靠;考虑数据来源的可信度以及是否有足够的证据支持作者的结论;识别作者在论
	文中可能存在的假设和偏见。考虑这些假设是否合理,以及它们如何影响论文的结论。考虑反
	驳和替代观点;审查研究方法的适当性和有效性;考虑研究设计是否合理,数据收集和分析是
	否准确;思考论文的结论在更广泛的学术和实际背景中的影响和应用;基于改论文你的可能的
	研究想法思路等等]}}
	
	
	
\end{document}
