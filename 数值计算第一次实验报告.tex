% This document is compiled using pdfLaTeX
% You can switch XeLaTeX/pdfLaTeX/LaTeX/LuaLaTeX in Settings

\documentclass[b5paper;twoside]{article}
\usepackage{ctex}
\usepackage{csquotes} % 对于智能引用符号
\usepackage[backend=biber,style=numeric]{biblatex} % 使用biblatex和biber
\usepackage{geometry}
\usepackage{graphicx}
\usepackage{listings}
\usepackage{minted}
\usepackage{amsmath}
\usepackage{amsfonts}
\usepackage[T1]{fontenc}
\usepackage{lmodern}
\usepackage{amssymb}
\usepackage{marginnote}
\usepackage{xcolor}
\usepackage{hyperref}
\usepackage{todonotes}
\usepackage[utf8]{inputenc}
\usepackage{mdframed}
\usepackage{multicol}
\usepackage{url} % 加载url包
\usepackage[linesnumbered,ruled,vlined]{algorithm2e}
\usepackage{array} % 提供>{...}功能
\usepackage{booktabs} % 提供更好的表格线条
\usepackage{lipsum} % 用于生成占位文本
\usepackage{titlesec}
\usepackage{titling}
\usepackage{wrapfig}
\usepackage{enumitem}
\usepackage{parskip}
\usepackage{graphicx}
\usepackage{tikz}
\usepackage{newtxmath}
\usepackage{caption}
\usepackage{minted}
\usepackage{fancyhdr}  % 用于设置页眉和页脚
\usepackage{tocloft}  % 用于自定义目录
\usepackage{fontspec}
\usepackage{titlesec} % 控制标题格式
\usepackage{setspace} % 控制行间距

% 设置页面布局
\pagestyle{fancy}
% 清除默认的页眉和页脚设置
\fancyhf{}
% 设置页脚
\fancyfoot[L]{苏睿熹}  % 左侧自定义文本
\fancyfoot[C]{数值计算第一次实验报告}  % 中间自定义文本
% 设置页脚线的长度
\renewcommand{\footrulewidth}{0.4pt}  % 设置页脚线的粗细
\renewcommand{\headwidth}{\textwidth} % 设置页眉线的长度为文本宽度
% 设置页眉
\fancyhead[L]{\rightmark}  % 显示当前小节名称
\fancyhead[R]{\thepage}    % 显示页码
% 设置页眉线的长度
\renewcommand{\headrulewidth}{0.4pt}  % 设置页眉线的粗细
\renewcommand{\headwidth}{\textwidth} % 设置页眉线的长度为文本宽度

% 设置正文字体:中文为宋体,英文为 Times New Roman,字号为小四
\renewcommand{\normalsize}{\fontsize{12bp}{15bp}\selectfont} % 小四字号,1.25倍行
%距
\linespread{1.25} % 行间距1.25倍

% 设置标题格式
\newcommand{\settitleformat}[2]{
	\titleformat{#1}[block]{\bfseries}{#2}{1em}{}
}
% 按等级设置标题字体大小
\titleformat{\section}{\fontsize{16pt}{20pt}\bfseries}{\thesection}{1em}{}
\titleformat{\subsection}{\fontsize{15pt}{18pt}\bfseries}{\thesubsection}{1em}{}
\titleformat{\subsubsection}{\fontsize{14pt}{17pt}\bfseries}{\thesubsubsection}{1em}{}
\setmainfont{Times New Roman}
\geometry{
    left=2cm,         % 左边距
    right=2cm,        % 右边距
    top=2.5cm,          % 上边距
    bottom=2.5cm        % 下边距
}
\lstset{
	language=Python,
	basicstyle=\ttfamily\small, % 设置代码的基本字体
	keywordstyle=\color{blue}, % 关键字的颜色
	stringstyle=\color{red},   % 字符串的颜色
	commentstyle=\color{gray}, % 注释的颜色
	numbers=left,              % 显示行号
	numberstyle=\tiny\color{gray}, % 行号的格式
	stepnumber=1,              % 行号的步长
	tabsize=4,                 % Tab 的宽度
	showstringspaces=false     % 去掉字符串中的空格下划线
}

\title{数值计算第一次实验报告}
\date{}

\begin{document}

\section{程序1}

\subsection{题目描述}

给定一个函数 $\Psi(x)$,该函数表示一个递归和的形式,形式如下:

\[
\Psi(x) = \sum_{k=1}^{\infty} \frac{1}{k(k + x)}
\]

需要实现一个算法来计算这个函数的近似值,要求截断误差在 $10^{-6}$ 内,给定一系列输入 
$x$,输出 $x$ 和 $\Psi(x)$ 的值。

\subsection{算法描述}

本程序使用逐项累加的方式计算 $\Psi(x)$ 的值,直到当前项的绝对值小于指定的误差阈值。算法
步骤如下:

\begin{enumerate}
	\item 初始化变量 $k=1$ 和 $\text{sum\_psi}=0.0$,分别表示和的当前项数和累计的 
	$\Psi(x)$ 值。
	\item 计算当前项值,$\text{term} = \frac{1}{k(k + x)}$,并将其加到 
	$\text{sum\_psi}$ 中。
	\item 如果当前项的绝对值大于误差阈值,继续计算下一项,更新 $k$ 值。
	\item 返回最终的 $\text{sum\_psi}$ 作为 $\Psi(x)$ 的值。
\end{enumerate}

\subsection{程序代码}

\begin{lstlisting}
import numpy as np

# 定义计算 Psi(x) 的函数
def compute_psi(x, error_threshold=1e-6):
k = 1  # 初始化项数
sum_psi = 0.0  # 初始化 Psi(x) 的值
term = 1.0  # 初始化项的值,确保第一次循环进入

# 当当前项的绝对值大于误差阈值时继续累加
while abs(term) > error_threshold:
term = 1 / (k * (k + x))  # 计算当前项
sum_psi += term  # 将当前项加到 Psi(x) 的和中
k += 1  # 增加项数

return sum_psi  # 返回计算的 Psi(x)

# 输入要计算的 x 值列表
x_values = [0.0, 0.5, 1.0, np.sqrt(2), 10.0, 100.0, 300.0]

# 对每个 x 值计算 Psi(x)
results = {x: compute_psi(x) for x in x_values}

# 输出结果
for x, psi in results.items():
print(f"x = {x:.6f}, Psi(x) = {psi:.6f}")
\end{lstlisting}

\subsection{程序输入输出}

输出:
\[
x = 0.000000, \Psi(x) = 1.643935
\]
\[
x = 0.500000, \Psi(x) = 1.226412
\]
\[
x = 1.000000, \Psi(x) = 0.999001
\]
\[
x = 1.414214, \Psi(x) = 0.873984
\]
\[
x = 10.000000, \Psi(x) = 0.291898
\]
\[
x = 100.000000, \Psi(x) = 0.050875
\]
\[
x = 300.000000, \Psi(x) = 0.019947
\]

\section{程序2}

\subsection{题目描述}

根据美国1920年到1970年的各年人口数据(见表A1),使用拉格朗日插值法计算1910年、1965年和2002
年的估计人口,并分析这些估计值的准确性。已知1910年实际人口为91772000。

\subsection{算法描述}

拉格朗日插值法是一种通过给定数据点来构造多项式的插值方法。对于给定的六个数据点(1920年
到1970年的年份和对应人口),可以通过以下的拉格朗日插值公式进行估算:

\[
P(x) = \sum_{i=0}^{5} y_i L_i(x)
\]

其中,

\[
L_i(x) = \prod_{j=0, j \neq i}^{5} \frac{x - x_j}{x_i - x_j}
\]

\begin{itemize}
	\item $x_i$ 和 $y_i$ 是已知数据点的年份和对应的 population 数据。
	\item $L_i(x)$ 是拉格朗日基函数。
\end{itemize}

对于目标年份(例如1910年、1965年、2002年),通过计算插值多项式 $P(x)$,可以得到这些年
份的估计人口。

此外,还需要计算误差系数来估算插值的准确性。误差系数的计算公式为:

\[
E(x) = \frac{P(x) - A(x)}{(x - x_0)(x - x_1)\cdots(x - x_5)}
\]

其中,$A(x)$ 是实际人口数据。

\subsection{程序代码}

\begin{lstlisting}
import numpy as np

# 拉格朗日插值的实现
def lagrange_interpolate(x_points, y_points, target_x):
'''进行拉格朗日插值计算'''
x0, x1, x2, x3, x4, x5 = x_points
y0, y1, y2, y3, y4, y5 = y_points

# 定义每个拉格朗日基函数
L0 = lambda x: (x - x1) * (x - x2) * (x - x3) * (x - x4) * (x - x5) / ((x0 - 
x1) * (x0 - x2) * (x0 - x3) * (x0 - x4) * (x0 - x5))
L1 = lambda x: (x - x0) * (x - x2) * (x - x3) * (x - x4) * (x - x5) / ((x1 - 
x0) * (x1 - x2) * (x1 - x3) * (x1 - x4) * (x1 - x5))
L2 = lambda x: (x - x0) * (x - x1) * (x - x3) * (x - x4) * (x - x5) / ((x2 - 
x0) * (x2 - x1) * (x2 - x3) * (x2 - x4) * (x2 - x5))
L3 = lambda x: (x - x0) * (x - x1) * (x - x2) * (x - x4) * (x - x5) / ((x3 - 
x0) * (x3 - x1) * (x3 - x2) * (x3 - x4) * (x3 - x5))
L4 = lambda x: (x - x0) * (x - x1) * (x - x2) * (x - x3) * (x - x5) / ((x4 - 
x0) * (x4 - x1) * (x4 - x2) * (x4 - x3) * (x4 - x5))
L5 = lambda x: (x - x0) * (x - x1) * (x - x2) * (x - x3) * (x - x4) / ((x5 - 
x0) * (x5 - x1) * (x5 - x2) * (x5 - x3) * (x5 - x4))

# 计算插值结果
return y0 * L0(target_x) + y1 * L1(target_x) + y2 * L2(target_x) + y3 * 
L3(target_x) + y4 * L4(target_x) + y5 * L5(target_x)

# 计算误差系数项
def calculate_error_coefficient(x_points, target_x, predicted_y, actual_y):
'''计算误差系数'''
x0, x1, x2, x3, x4, x5 = x_points
delta_y = actual_y - predicted_y
return delta_y / ((target_x - x0) * (target_x - x1) * (target_x - x2) * 
(target_x - x3) * (target_x - x4) * (target_x - x5))

# 计算误差
def compute_error(x_points, target_x, error_coefficient):
'''计算指定年份的误差'''
x0, x1, x2, x3, x4, x5 = x_points
return error_coefficient * (target_x - x0) * (target_x - x1) * (target_x - x2) 
* (target_x - x3) * (target_x - x4) * (target_x - x5)

# 数据(年份和对应人口数据)
years = [1920, 1930, 1940, 1950, 1960, 1970]
populations = [105711, 123203, 131669, 150697, 179323, 203212]

# 目标年份:1910、1965、2002
target_years = [1910, 1965, 2002]

# 使用拉格朗日插值法估算这些年份的人口
estimated_populations = [lagrange_interpolate(years, populations, year) for 
year in target_years]

# 输出估算结果
for target_year, estimated_population in zip(target_years, 
estimated_populations):
print(f"预估{target_year}年人口: {estimated_population:.0f} (千人)")

# 实际人口数据
actual_population_1910 = 91772  # 1910年实际人口(千人)

# 计算误差系数
error_coefficient = calculate_error_coefficient(years, target_years[0], 
estimated_populations[0], actual_population_1910)

# 计算1965年和2002年的误差
error_1965 = compute_error(years, target_years[1], error_coefficient)
error_2002 = compute_error(years, target_years[2], error_coefficient)

# 输出误差
print(f"1965年误差: {error_1965:.0f} (千人)")
print(f"2002年误差: {error_2002:.0f} (千人)")

\end{lstlisting}

\subsection{程序输入输出}

输出:
\[
\text{预估1910年人口: } 31872 \ (\text{千人})
\]
\[
\text{预估1965年人口: } 193082 \ (\text{千人})
\]
\[
\text{预估2002年人口: } 26139 \ (\text{千人})
\]
\[
\text{1965年误差: } -1228 \ (\text{千人})
\]
\[
\text{2002年误差: } 2128310 \ (\text{千人})
\]


\section{程序3}

\subsection{题目描述}

根据表A1的数据点(年份,人口/千人): 
\[
((x_0, y_0),(x_1, y_1), \dots, (x_n, y_n))
\]
通过牛顿插值法估计1965年与2012年的人口数。

\subsection{算法描述}

牛顿插值法通过差商表逐步计算插值多项式,利用已知数据点计算目标点的估算值。其算法过程如下:

\begin{enumerate}
	\item 计算差商表:
	\[
	f[x_0] = y_0
	\]
	\[
	f[x_1, x_0] = \frac{f[x_1] - f[x_0]}{x_1 - x_0}
	\]
	\[
	f[x_2, x_1, x_0] = \frac{f[x_2] - f[x_1]}{x_2 - x_1} - \frac{f[x_1] - 
	f[x_0]}{x_1 - x_0}
	\]
	以此类推,计算差商表。
	
	\item 使用牛顿插值公式计算目标点的估算值:
	\[
	P(x) = f[x_0] + (x - x_0)f[x_1, x_0] + (x - x_0)(x - x_1)f[x_2, x_1, x_0] + 
	\dots
	\]
\end{enumerate}

\subsection{程序代码}

\begin{lstlisting}

import numpy as np
def newton_interpolation(X, Y, x):
'''牛顿插值法实现'''
n = len(X) 
# 初始化差商表
f = np.zeros((n, n))  # 使用NumPy来初始化差商表
for i in range(n):
f[i][0] = Y[i]  # 第一列是Y值,即原始人口数据

# 计算差商
for j in range(1, n):
for i in range(n - j):
f[i][j] = (f[i + 1][j - 1] - f[i][j - 1]) / (X[i + j] - X[i])

# 使用牛顿插值公式计算目标年份的估算人口
result = f[0][n - 1]
for i in range(n - 2, -1, -1):
result = f[0][i] + (x - X[i]) * result

return result

# 给定的年份和对应的人口数据
X = np.array([1920, 1930, 1940, 1950, 1960, 1970])
Y = np.array([105711, 123203, 131669, 150697, 179323, 203212])

# 估算1965年和2002年的人口
x0, x1 = 1965, 2002
y0, y1 = newton_interpolation(X, Y, x0), newton_interpolation(X, Y, x1)

# 输出结果
print(f"1965年估算人口: {y0:.0f} (千人)")
print(f"2002年估算人口: {y1:.0f} (千人)")

\end{lstlisting}

\subsection{程序输入输出}

输出:
\[
\text{1965年估算人口: } 193082 \ (\text{千人})
\]
\[
\text{2002年估算人口: } 26139 \ (\text{千人})
\]

\section{程序7}

\subsection{题目描述}

给定 $n+1$ 个插值点和一阶导数的端点值 ($m_0$, $m_n$),用三次样条插值法构造函数 
$S(x)$,并求在给定点 $x$ 处 $S(x)$ 的值。

\subsection{算法描述}

本题使用\textbf{三次样条插值}方法来计算函数值。具体地,三次样条插值法构造一个连续的三次
多项式来拟合数据,并且要求在端点处满足一阶导数条件。对于给定的插值点 ($x_i, 
f(x_i)$),$i = 0, 1, 2, \dots, n$,以及端点的导数值 $m_0$ 和 $m_n$,我们使用三次样
条插值的数学表达式:

\[
S(x) = 
\begin{cases} 
	f_0 + m_0 (x - x_0) + \frac{(x - x_0)^2}{2} \left( f''_0 + \frac{f''_0 - 
	f''_1}{x_1 - x_0} \right), & x_0 \leq x \leq x_1 \\
\end{cases}
\]

其中,所有中间区间 $S(x)$ 由类似的三次多项式构成,并且满足平滑性和导数连续性。

\subsection{程序代码}

\begin{lstlisting}
import numpy as np
from scipy.interpolate import CubicSpline

def cubic_spline_interpolation(x_values, y_values, m_0, m_n, x_eval):
"""
使用三次样条插值并考虑边界条件(端点导数)。

参数:
x_values (array-like): 插值点的 x 值
y_values (array-like): 插值点的 y 值
m_0 (float): 在 x_0 处的导数值
m_n (float): 在 x_n 处的导数值
x_eval (float): 要计算 S(x) 的 x 值

返回:
float: 在 x_eval 处的 S(x) 的值
"""
# 构造三次样条插值,考虑边界导数条件
cs = CubicSpline(x_values, y_values, bc_type=((1, m_0), (1, m_n)))

# 计算 x_eval 处的 S(x) 的值
return cs(x_eval)

if __name__ == "__main__":
# 输入插值点数 n
n = int(input("插值点数:"))

# 输入插值点
x_values = []
y_values = []
for _ in range(n + 1):
x, y = map(float, input("x y:").split())
x_values.append(x)
y_values.append(y)

# 输入一阶导数的端点值
m_0 = float(input("在 x_0 处的导数值 m_0: "))
m_n = float(input("在 x_n 处的导数值 m_n: "))

# 输入要计算的函数点 x_eval
x_eval = float(input("要计算的函数点: "))

# 调用函数进行计算
s_x = cubic_spline_interpolation(x_values, y_values, m_0, m_n, x_eval)

# 输出结果
print(f"x = {x_eval} 处的 S(x) 的值为: {s_x:.6f}")

\end{lstlisting}

\subsection{程序输入输出}

示例输入:
\[
\text{插值点数:3}
\]
\[
\text{x y:}
\]
\[
\begin{array}{cc}
	0 & 0 \\
	1 & 1 \\
	2 & 0 \\
	3 & -1 \\
\end{array}
\]
\[
\text{在 } x_0 \text{ 处的导数值 } m_0: 1
\]
\[
\text{在 } x_n \text{ 处的导数值 } m_n: -1
\]
\[
\text{要计算的函数点: } 1.5
\]

示例输出:
\[
\text{x = 1.5 处的 } S(x) \text{ 的值为: } 0.666667
\]


\section{程序10}

\subsection{题目描述}

使用 Newton 迭代法求解非线性方程组:

\[
\begin{cases}
	f(x) = x^2 + y^2 - 1 = 0 \\
	g(x) = x^3 - y = 0
\end{cases}
\]

给定初始点 $(x_0, y_0) = (0.8, 0.6)$,使用误差控制条件:

\[
\max(|x_k - x_{k-1}|, |y_k - y_{k-1}|) \leq 10^{-5}
\]

\subsection{算法描述}

\begin{enumerate}
	\item \textbf{初始化}:选择初始猜测点 $(x_0, y_0)$,并设定容许的误差阈值。
	\item \textbf{迭代过程}:使用 Newton 迭代法的公式:
	\[
	\mathbf{J}(x_k, y_k) \cdot \mathbf{\Delta} = -\mathbf{F}(x_k, y_k)
	\]
	其中,$\mathbf{J}(x_k, y_k)$ 是雅可比矩阵,$\mathbf{F}(x_k, y_k)$ 是方程组的函
	数值,$\mathbf{\Delta} = (\Delta x_k, \Delta y_k)$ 是更新量。
	\item \textbf{更新}:根据迭代结果更新 $x_k$ 和 $y_k$,直到满足误差控制条件或达到
	最大迭代次数。
	\item \textbf{收敛性检查}:当 $\max(|\Delta x_k|, |\Delta y_k|)$ 小于给定的误
	差阈值时,停止迭代,输出结果。
\end{enumerate}

\subsection{程序代码}

\begin{lstlisting}
import numpy as np

# 定义方程 f 和 g
def f(x, y):
return x**2 + y**2 - 1

def g(x, y):
return x**3 - y

# 定义雅可比矩阵
def jacobian(x, y):
return np.array([[2 * x, 2 * y],
[3 * x**2, -1]])

# Newton 迭代法
def newton_method(x0, y0, tol=1e-5, max_iter=100):
x, y = x0, y0
for i in range(max_iter):
# 计算函数值和雅可比矩阵
F = np.array([f(x, y), g(x, y)])
J = jacobian(x, y)

# 解线性方程 J * delta = -F,求出 delta
delta = np.linalg.solve(J, -F)

# 更新 x 和 y
x += delta[0]
y += delta[1]

# 打印当前迭代步的结果
print(f"Iteration {i+1}: x = {x}, y = {y}")

# 检查收敛条件
if max(abs(delta[0]), abs(delta[1])) < tol:
print("Converged.")
return i + 1, x, y

print("Maximum iterations reached without convergence.")
return max_iter, x, y

if __name__ == "__main__":
# 输入初始点 (x0, y0) 和精度控制值 tol
x0, y0 = map(float, input("输入初始点 (x0, y0):").split())
tol = float(input("输入精度控制值 e:"))

# 调用 Newton 方法求解
k, x, y = newton_method(x0, y0, tol)

# 输出结果
print(f"迭代次数 k = {k}")
print(f"第 {k} 步的迭代解: x = {x:.6f}, y = {y:.6f}")

\end{lstlisting}

\subsection{程序输入输出}

输入初始点 \( (x_0, y_0) \):\( 0.8, 0.6 \)  
输入精度控制值 \( e \):\( 1e-5 \)  

迭代 1: \( x = 0.8270491803278689, y = 0.5639344262295083 \)  \\
迭代 2: \( x = 0.8260323731676462, y = 0.5636236767037873 \)  \\
迭代 3: \( x = 0.8260313576552345, y = 0.5636241621608473 \)  

收敛。  

迭代次数 \( k = 3 \)  

第 3 步的迭代解: \( x = 0.826031, y = 0.563624 \)


\end{document}
